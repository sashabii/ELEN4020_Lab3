\documentclass[twocolumn]{IEEEtran}


\usepackage{algorithm}
\usepackage{algorithmic}
\usepackage{url}

\usepackage{graphicx}
\usepackage{caption}
\usepackage[margin=0.7in,a4paper]{geometry}
\graphicspath{{images/}}

\title{ELEN4020A: Data Intensive Computing Lab 3}
\author{Jared Gautier (820687), Nick Raal (793528), Sasha Berkowitz (818737), Arunima Pathania(1117426)}
%\IEEEauthorblockN{School of Electrical and Information Engineering, University of the Witwatersrand} \ {ELEN4020: Data Intensive Computing}	


\begin{document}
	\maketitle
	
	\section{Introduction}
	The objective of the lab was to perform matrix multipication using various frameworks and MapReduce. The MapReduce framework was introduced by Google to support distributed computing on large data sets onto clusters of computers. The data is replicated multiple times in parallel on the system for increased efficiency, reliability and availability. This was done by using the MrJob framework adapted to two different algorithms in the the Python language.
	
	
	\section{Algorithms}
	The reason for using the MapReduce framework is due to the benefits it provides. MapReduce has the ability to take a query over a data set, divide it, then run the query in parallel over multiple nodes. This benedits in removing the issue of small computers processing data too large to handle, using multiple servers and the Batch processing model.
	
	\subsection{Map}
	The main purpose of the map function is to generate a key,value pair.  The map function takes individual tasks and transforms the input records into intermediate records, which can be processed by the multiplication algorithm. Each of these transformations occur in parallel. The map function shuffles the key, value pairs based on the first key to re-organize the output. 
	
	
	
	\subsection{Reduce}
	The reduce function takes the re-organized data from the map function and reduces them to a summarized data set, the desired output. The reduce function, in terms of this laboratory, performs matrix multiplication using the generated key,value pair.This is done by multiplying the values with the keys and storing them. The final value is the sum of the different products obtained in the previous step.
	
	\subsection{Algorithm A}
	The first algorithm uses the MrJobs MapReduce framework.
	\subsubsection{Mapper}
	 The mapping function, \textit{mapFn} shown in algorithm 1, takes the input matrix \textit{.txt} files and generates the key, value pairs in terms of columns for matrix A, and the key, value pairs in terms of rows for matrix B.\\
	
	\subsubsection{Reducer}
	 From the generated pairs, the reduce function, \textit{reduceFn} show in algorithm 2, appends the key, value pairs into an array, then multiplies the matricies. This multiplied value  is then added using the \textit{addition} function to get the final result. The speed of this algorithm would be observed to be $O(n^{3})$.
		
		\begin{algorithm}
			\caption{mapFn(\_ , line)}
			\begin{algorithmic} 
		
				\REQUIRE {Mapping function to generate key, value pairs}
				\STATE $row, col, value \leftarrow line.split$
				
				\STATE$filename \leftarrow os.environ["input file"] $
				
				\IF{filename = "matrix1.txt"}
					\STATE{$yield$ $col , (0,row,value)$}
					\ELSIF{filename = matrix2.txt}
						\STATE{$yield$ $row,(1,col,value)$}
				\ENDIF
				
			\end{algorithmic}
		\end{algorithm}
		
		
		
		

		
		
		
		
		\begin{algorithm}
			\caption{reduceFn(y,value)}
			\begin{algorithmic} 
				\REQUIRE {Uses output of mapFn function to perform matrix multiplication}
				
				\STATE{rowVals[]}
				\STATE{colVals[]}
				
				\FOR{$x \leftarrow 0$ to $values$}
				
				\IF{x[0] = 0}
				\STATE{rowVals.append(x)}
				\ENDIF
				\IF{x[0] = 1}
				\STATE{colVals.append(x)}
				\ENDIF
				
				\FOR{$a,b,row \leftarrow 0$ to $rowVals$}
				\FOR{$a,key,col \leftarrow 0$ to $colVals$}
				\STATE{$yield$ $(b,key) , (int(row)*int(col))$}
				
				\ENDFOR
				\ENDFOR
			
				\ENDFOR		
				
			\end{algorithmic}
		\end{algorithm}
		
	
	
	
	
	

	\begin{algorithm}
		\caption{The mapper function}
		\begin{algorithmic} 
			\REQUIRE {Map function to produce and return key, value pairs}
			\STATE
			
			\FOR{$value\_ A \leftarrow 0$ to $A$}
			\STATE $k \leftarrow 1$ to $ B $
			\STATE $((i,k), (A, j, value\_ A))$ for each value of $k$
			\ENDFOR
			
			\FOR{$value\_ B \leftarrow 0$ to $B$}
			\STATE $i \leftarrow 1$ to $ A $
			\STATE $((i,k), (B, j, value\_ B))$ for each value of $i$
			
			\ENDFOR		
			
			
			\RETURN $(key, value)$ pair
			\end{algorithmic}
			\end{algorithm}
			
	\subsection{Algorithm B}
	Algorithm B also uses the MrJob framework in python. The mapper function is very similar to the mapper function of algorithm A, this is because the mapper is used for generating the key value pairs. The change between the two algorithm is seen in the reduce functions where different implementations are used.
	
		
	\subsubsection{Reducer}
	The reducer function implemented can be seen in Algorithm 4. This algoritm employes a different way of reduction. This algorithm is more efficient as the number of for loops has been limited. This change brings the time complexity down to $O(n^{2})$. The implementation of algorithm B uses the \textit{sum()} function to add the mulitplied matrix results together.
	
	
		
		\begin{algorithm}
			\caption{The Reducer Function}
			\begin{algorithmic} 
				\REQUIRE {Uses output of mapFn function to perform matrix multiplication}
				
				\STATE{rowVals[]}
				\STATE{colVals[]}
				
				\FOR{$x \leftarrow 0$ to $values$}
				
				\IF{x[0] = 0}
				\STATE{rowVals.append(x)}
				\ENDIF
				\IF{x[0] = 1}
				\STATE{colVals.append(x)}
				\ENDIF
				
				\ENDFOR
				\FOR{$i \leftarrow 0$ to $j$}
				\STATE $Ans.append(rowVals[i]*colVals[i])$
				\ENDFOR
				
				\STATE yield $ k,sum(Ans)$
				
			
			\end{algorithmic}
		\end{algorithm}
	
	\section{Results}
	The results for each of the algorithms are shown below in table 1. From the table, it can be deduced that Algorithm B was more efficient in completing the calculations. This could be due to the \texttt{ReduceFn} function in Algorithm B having time complexity of $O(n)$, as opposed to the time complexity of $O{n^2}$ in Algorithm A.
	
\begin{table}[!h]
	\centering
	\caption{Table showing completion times of each algorithm to multiple the matricies}
	\label{my-label}
	\begin{tabular}{|l|l|l|ll}
		\cline{1-3}
		\textbf{Size of matrix} & \textbf{Algorithm A} & \textbf{Algorithm B} &  &  \\ \cline{1-3}
		10 x 10                 & 1.226556             & 0.84565                    &  &  \\ \cline{1-3}
		100 x 100               & 20.135463            & 11.35642                    &  &  \\ \cline{1-3}
		1000 x 1000             & c                    & d                    &  &  \\ \cline{1-3}
	\end{tabular}
\end{table}

The algorithm were tested using the Anaconda application on a Windows machine. It is assumed that delays were experienced as a result of this and the algorithms would run faster using the Linux environment.
	
	
	\section{Unweighted Directed Graphs}
	Directed graphs involve a series of interconnected nodes, the nodes are connected by edges. In order to calculate all of the groups of paths of length 3, an adjacency matrix data structure must be implemented. Where the adjacency matrix is given by a matrix of ones and zeroes, where ones indicate an edge between a vertex and 0s represent no connection. This can be implemented by means of an adjacency list data structure, which is a list of interconnected vertices that is similar to how a 2D array would be represented using data structures \cite{graph}. The preferable data structure would be the data list, this is because the space of the list would be 0(n+m), as opposed to the adjacency matrix which has a space  complexity of $0(n^{2})$.
	Once this data structure is implemented, finding the paths of length three would be searching the list for the links that have a length greater than or equal to three.


\section{Conclusion}
Algorithm B the desirable implementation for matrix multiplication because it was seen to be faster , this can be seen from the results section. The MapReduce framework was successfully implemented, this shows the parallelism improves the efficiency of any desired function. A solution to finding weights of three was found for directed graphs, this was done by using data lists.




	\begin{thebibliography}{12}
		\bibitem{graph}
		Author unknown. 2015. \textit{Directed and Undirected Graphs}. [ONLINE] Available at: \url{http://www.inf.ed.ac.uk/teaching/courses/cs2/LectureNotes/CS2Bh/ADS/lecture9.pdf} [Accessed 5 April 2018].
	\end{thebibliography}

\end{document}

